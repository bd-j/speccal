%\begin{figure}[h!]
%\begin{center}
%\includegraphics[width = 0.5 \textwidth]{figures/data.pdf}
%\caption{Observational data. 
%Top: The calibrated spectrum for \excluster.
%Middle: The signal-to-noise ratio.
%Bottom: The spectrophotometric calibration vector determined by
%\citet{schiavon05}. Note the wiggles in the calibration vector\label{fig:ggc_data}}
%\end{center}
%\end{figure}


\begin{figure*}[h!]
\includegraphics[width=0.5 \textwidth]{figures/ideal.pdf}
\caption{Results of inference from a mock spectrum and a single
  photometric data point where the spectrophotometric calibration is
  assumed to be perfectly known. 
Top: Model spectra ({\it green}) and photometry ({\it orange})
constructed from draws from the posterior PDFs of the model
parameters, compared to the true intrinsic spectroscopy and photomtry
({\it black})
Bottom: Marginalized posterior PDFs for several of the physical
parameters of interest.  The input mock parameters are shown as
vertical black lines.
{\color{red} Show residuals.}
\label{fig:speconly_ideal}}
\end{figure*}


\begin{figure*}[h!]
\includegraphics[width=\textwidth]{figures/speconly_calibrated.pdf}
\caption{Results of inference from a \emph{perfectly calibrated} mock
  spectrum and a single photometric data point, where uncertainties in
  the calibration are modeled.
Top Left: The uncertainty normalized residuals between the mock
observed spectrum ({\it black}) and YY model spectra constructed from
draws from the posterior PDFs of the model parameters, including
calibration parameters {\it blue}.
Bottom Left: In this case the true calibration vector ({\it black}) is a constant.
Posterior samples of the inferred calibration vector ({\it blue}) include both
the polynomial and the Gaussian Process mean prediction (Equation
\ref{eq:calibration}).
Top Right: Samples of the posterior prediction for the
\emph{intrinsic} spectrum ({\it blue}) as well as for the photometry
({\it orange}).  The true mock photometry is also shown ({\it
black}).
Bottom Right: Marginalized posterior PDFs for several of the physical
parameters of interest.  The input mock parameters are shown as
vertical lines.
 {\color{red} This is not converged.  mass and dust2 PDFs will become
   much broader.}
\label{fig:speconly_calibrated}}
\end{figure*}

\begin{figure*}[h!]
\includegraphics[width=\textwidth]{figures/speconly_uncalibrated.pdf}
\caption{Results of inference from an \emph{uncalibrated} mock
  spectrum and a single photometric data point.  Panels and colors are
  as in Figure \ref{fig:speconly_calibrated}.  In this case the
  observed spectrum is in arbitrary units and the true
  calibration vector is not a constant with wavelength.
\label{fig:speconly_uncalibrated}}
\end{figure*}


\begin{figure*}[h!]
\includegraphics[width=0.5 \textwidth]{figures/photonly.pdf}
\caption{Results of inference from the photometric data only.  The
  redshift is fixed {\color{blue} (should just use a tight prior)} and
 spectroscopic calibration parameters are not modeled.
Top: The intrinsic spectrum. Samples of the posterior prediction
for the \emph{intrinsic} spectrum ({\it red}) and the
photometry ({\it orange}).  The mock photometry is also shown
({\it black}).
Right: Marginalized posterior PDFs for several of the physical
parameters of interest.  The input mock parameters are shown as
vertical lines.
\label{fig:photonly}}
\end{figure*}

\begin{figure*}[h!]
\includegraphics[width=\textwidth]{figures/specphot_uncalibrated.pdf}
\caption{Results of inference from the combination of
\emph{uncalibrated} spectroscopy and all the photometric data points.
Panels and are as in Figure \ref{fig:speconly_calibrated}.
\label{fig:specphot_uncalibrated}}
\end{figure*}



\begin{figure*}[h!]
\includegraphics[width=\textwidth]{figures/components.pdf}
\caption{Components of the spectroscopic calibration.  
Top Left: Posterior samples of the smooth component of the calibration
vector, the exponential of a polynomial.  The input true calibration
vector is shown in {\it black}.
Bottom Left: The Gaussian Process prediction conditioned on the residual
between the data and posterior samples of the mean model including the
smooth calibration component.
Top Right: Posterior samples of the sum of the smooth calibration
component and the Gaussian Process prediction, compared to the input
true calibration vector ({\it black}).
Bottom Right: Ratio of the sum of the smooth calibration component and
the Gaussian Process prediction to the input
calibration vector ({\it black}).
\label{fig:components}}
\end{figure*}


\begin{figure*}[h!]
\includegraphics[width=\textwidth]{figures/calibration_post.pdf}
\caption{Joint Posterior PDFs for the calibration parameters inferred
from the combination of \emph{uncalibrated} spectroscopy and all the
photometric data points.
\label{fig:cal_posterior}}
\end{figure*}


\begin{figure*}[h!]
\includegraphics[width=\textwidth]{figures/combined_post.pdf}
\caption{Corner plot of joint and marginal posterior PDFs for several
physical parameters. The posteriors obtained from photometry only are
in {\it red}, from \emph{uncalibrated} spectroscopy and a single
photometric point in {\it blue}, and from the photometry and
spectroscopy combined in {\it magenta}.  The contours are 1 and
2-sigma contours. The true values are indicated by {\it black points}.
\label{fig:combined_posterior}}
\end{figure*}


\begin{figure*}[h!]
\includegraphics[width=\textwidth]{figures/noise_realizations.pdf}
\caption{Percentiles of the posterior PDFs for four of the parameters,
obtained from multiple noise realizations of the mock photometry and
uncalibrated spectra.  The input mock parameters are shown as a black
horizontal line, The green shaded region gives the 16th to 84th
percentile range for a noiseless mock.
\label{fig:noise_realizations}}
\end{figure*}


\begin{figure*}[h!]
\includegraphics[width=\textwidth]{figures/vary_params.pdf}
\caption{Offset of the posteriors from the input parameters for
inference from mock spectra and photometry, as a function of the mock
parameter that is varied.  The medians of the posterior PDFs are shown
as connected black circles, the 16th and 84th percentile of the
posteriors are indicated by the gray shaded region.
Left: Results when varying age.
Middle: Results when varying dust attenuation.
Right: Results when varying metallicity.
 \label{fig:mock_parameter_space}}
\end{figure*}


\begin{figure*}[h!]
\includegraphics[width=\textwidth]{figures/vary_nphot.pdf}
\caption{The value of more photometry in combination with optical
spectroscopy. Marginalized posterior PDFs obtained from mock spectra
and photometry, for different numbers of photometric bands. The
medians of the posterior PDFs are shown as connected black circles,
the 16th and 84th percentile of the posteriors are indicated by the
gray shaded region.
\label{fig:vary_phot}}
\end{figure*}


\begin{figure*}[h!]
\includegraphics[width=\textwidth]{figures/real_calibrated.pdf}
\caption{Inference from a real, \emph{calibrated}, spectrum. 
 Top left: Residuals.  
Bottom left: Inferred calibration vactor.  
Top right: Inferred instrinsic SED.
Bottom right: posterior PDFs for selected physical parameters.
\label{fig:real_calibrated}}
\end{figure*}

\begin{figure*}[h!]
\includegraphics[width=\textwidth]{figures/real_uncalibrated.pdf}
\caption{Inference from a real, \emph{uncalibrated}, spectrum.   
Top left: Residuals.  
Bottom left: Inferred calibration vactor.  
Top right: Inferred instrinsic SED.
Bottom right: posterior PDFs for selected physical parameters.
\label{fig:real_uncalibrated}}
\end{figure*}

%\begin{figure*}[h!]
%\includegraphics[width=\textwidth]{figures/inferred_calibration.pdf}
%\caption{Left: Ratio of the calibration vector inferred from the
%  real uncalibrated spectrum to that inferred from the real calibrated
%  spectrum ({\it green}). Taking the ratio reduces the effects of model
%  imperfections and shows how well we are able to infer the
%  calibration vector of \citet{schiavon05} ({\it black}).  
%Right: Joint posterior PDFs for the inferred calibration parameters.
%\label{fig:real_calibration_ratio}}
%\end{figure*}

\begin{figure*}[h!]
\includegraphics[width=\textwidth]{figures/real_post.pdf}
\caption{Inferred parameters for \excluster.  We show the inference from the
  calibrated spectrum ({\it green}), from the uncalibrated spectrum
  ({\it orange}), and from the CMD ({\it black}).
\label{fig:real_parameters}}
\end{figure*}

\documentclass[12pt, preprint]{aastex}

\input{vc}

\begin{document}

\title{Combining information from photometric and spectroscopic data:
  Don't waste your time in spectral calibration!}
\author{BDJ, DRW, DFM, DWH, CC, others}

\begin{abstract}
In a typical spectral-fitting project,
  there tend to be a small number of bands of well-calibrated photometry
  and thousands of poorly calibrated spectroscopic pixels.
Good spectroscopic calibration is both expensive and difficult,
  but naive weighted least-square fitting will weight
  the thousands of spectroscopic pixels far higher overall
  than the few photometric bands.
\end{abstract}

\keywords{
  greetings
  ---
  whole world
}

All the code used in this project is available under an open-source license
  from \url{http://github.com/bd-j/something/}.
This version of the paper was generated
  from a git repository available at \url{http://github.com/bd-j/speccal/}
  with git hash \texttt{\githash} (\gitdate).

\acknowledgements
Thanks in advance to X. for bringing us beer.

\end{document}

% This file is part of the `speccal` project.
% Copyright 2014 the authors.
% All rights reserved.

% ## style notes:
% - all equations in `eqnarray` environments
% - parameters are `inferred`; `measurement` is reserved for data
% - use emulateapj or aastex, as you like, but watch for equation-wrapping in the former

% ---- Hogg (aastex) mode ----
\documentclass[12pt, letterpaper, preprint]{aastex}
% ----

% ---- Charlie (emulateapj) mode ----
% \documentclass[iop,numberedappendix]{emulateapj}
% \usepackage{apjfonts}
% ----

\bibliographystyle{apj}
\input{vc}
\newcommand{\transpose}[1]{{#1}^{\!\mathsf T}}
\newcommand{\given}{\,|\,}

\newcommand{\lick}{1}
\newcommand{\ccpp}{2}
\newcommand{\cds}{3}
\newcommand{\mpia}{4}

\begin{document}

\title{Combining information from photometric and spectroscopic data:\\
  Don't waste your time in spectrophotometric calibration!}
\author{
  BDJ\altaffilmark{\lick},
  DRW\altaffilmark{\lick},
  DFM\altaffilmark{\ccpp},
  DWH\altaffilmark{\ccpp, \cds, \mpia},
  CC\altaffilmark{\lick},
  others}
\altaffiltext{\lick}{University of California Lick Observatory}
\altaffiltext{\ccpp}{Center for Cosmology and Particle Physics, Department of Physics, New York University}
\altaffiltext{\cds}{Center for Data Science, New York University}
\altaffiltext{\mpia}{Max-Planck-Institut f\"ur Astronomie}

\begin{abstract}
In a typical spectral-fitting project,
  there tend to be a small number of bands of well-calibrated photometry
  and thousands of poorly calibrated spectroscopic pixels.
Good spectroscopic calibration is both expensive and difficult,
  but naive weighted least-square fitting will weight
  the thousands of spectroscopic pixels far higher overall
  than the few photometric bands.
Here we present a flexible and general spectroscopic calibration model,
  which can be fit simultaneously along with whatever spectral parameters.
The model involves a polynommial calibration vector to capture large-scale calibration issues
  and a Gaussian Process to capture small-scale wiggles.
We show that, in this framework, the quality of some kinds of spectral parameter fits
  is not a strong function of the quality of the spectroscopic calibration;
  that is, for many scientific goals there is no scientific reason
  to obtain good---or really any---spectrophotometric calibration.
In particular, for stellar population fitting, high signal-to-noise data,
  and good photometry,
  we show that raw counts are just as good as calibrated counts for making inferences about the
  parameters of interest.
This obviation of calibration overheads will pay all yall \emph{mad cheddar},
  at least if all yall employ full cost accounting.
\end{abstract}

\keywords{
  greetings
  ---
  whole world
}

\section{Introduction}

In probabilistic inference, all information flows from the data to the
parameters of interest by means of a likelihood function, or an
expression for the probability of the data conditioned on the
parameters.
If there are $N$ independent sources of data, for each of which there
is a likelihood function, the full-data likelihood is the product of
the $N$ individual likelihoods (or the log-likelihood is a sum of the
$N$ individual log-likelihoods).
There is no bespoke ``weighting'' of the data that is justifiable in
this framework.
This creates some apparent paradoxes!
One is the following:

Imagine you have three photometric bands through which you have
observed some object (independently, say), and also a 2500-pixel
spectrum, where the pixels are defined such that they are precisely or
approximately independent measurements.
For these purposes, two spectral pixels are very close to independent
measurements if they are derived from disjoint sets of detector pixels
in the spectrograph detector plane.
Each photometric band and each spectroscopic pixel has a
log-likelihood function, and these 2503 log-likelihood functions must
be (by the Rules of Righteousness) coadded equally into one full-data
log-likelihood.
The log-likelihood is therefore fully dominated by the spectral
pixels!
This despite the fact that in almost all astronomical investigations,
the photometry is far more reliable, per measurement, than the
spectra.
That is, if the spectral pixels tend to prefer one model, or one set
of parameters, and the photometry another, the spectra will almost
always ``win'' just by their overwhelming force.

What is a Righteous Probabilistic Investigator (RPI) to do?
One option---fully unjustified---is to ``down-weight'' the
log-likelihoods of the spectral pixels relative to the photometric
measurements.
This is equivalent to taking the likelihoods to some power, which has
no justification in inference (though it is sometimes used in
optimization and sampling as an ``annealing'' move).
This down-weighting is not justified, but there is even more
ad-hockery than this, because there is no principled way to choose the
unprincipled weights.

The Right Thing To Do (RTTD) is to express the \emph{reason} that the
RPI believes the spectroscopic data to be less reliable \emph{within}
the likelihood function.
That is, the RPI must add nuisance parameters to parameterize the
unreliability of the spectroscopy, and infer those along with the
parameters of interest.
If this parameterization is good, the information in the spectroscopy
data will not flow entirely to the parameters of interest; instead,
some will flow to the nuisance parameters.
This will reduce the importance of the spectroscopy relative to the
photometry and the RPI will be well pleased with the dominance of the
photometric data in the final inferences of the parameters of
interest.
In this what follows, we provide an example of this kind of
parameterization and inference.

Our example involves instantiating a ``calibration adjustment'' model,
which itself has both a polynomial component and a non-parametric
component.
The parameters of the calibration adjustment are nuisances, from a
scientific standpoint.
However, because the calibration adjustment is inferred along with the
parameters of interest, the probabilistic inference that provides
parameter estimates \emph{also} provides a spectrophotometric
calibration of the spectral data.
That is, the method we present has the potential to \emph{obviate
spectrophotometric calibration}.

We find that this potential---obviation of calibration---is realized.
That is, if the procedures demonstrated below become widely adopted,
some or perhaps most astronomical spectroscopic projects could
potentially drop or dramatically reduce their spectrophotometric
calibration programs.
This would reduce telescope overheads and data-analysis time and
complexity, delivering economic and sanity benefits to the whole
community.

\section{Model generalities}

THIS SECTION NEEDS WORK; THIS IS JUST AN EQUATION DUMP.

HOGG: A model is a parameterized likelihood function, with prior PDFs on the nuisance parameters (at least)...

HOGG: There is a likelihood for the photometry and for the spectroscopy...

HOGG: Each of these penalizes the difference between a column vector of data and a column vector of model expectation value...

The standard chi-squared or log-Gaussian log-likelihood is
\begin{eqnarray}
\ln p(d\given\theta) &=& -\frac{1}{2}\,\sum_{n=1}^N \left[\frac{[d_n - \mu_n(\theta)]^2}{\sigma_n^2} + \ln(2\pi\,\sigma_n^2) \right]
\quad ,
\end{eqnarray}
where $\theta$ is a blob of parameters of interest,
$d$ is the full set (in the form of a column vector) of data,
made up of $N$ values $d_n$,
$\mu(\theta)$ is the the column-vector mean prediction (expectation value) of the model,
made up of $N$ values $\mu_n(\theta)$,
the $\sigma_n^2$ are the observational noise variances.
We have included the $\ln(\sigma_n^2)$ term in the likelihood because
in what follows we are going to be modifying the noise model.

The calibration-vector and Gaussian-Process modified log-likelihood is
\begin{eqnarray}
\ln p(d\given\theta,\alpha) &=& -\frac{1}{2}\,\left[\transpose{\Delta}\,C^{-1}\,\Delta + \ln([2\pi]^N\,\det{C}) \right]
\\
\Delta &\equiv& d - f(\alpha)\ast\mu(\theta)
\\
C_{nn'} &\equiv& \sigma_n^2\,\delta_{nn'} + K_\alpha(\lambda_n - \lambda_{n'})
\quad ,
\end{eqnarray}
where $\alpha$ is a blob of nuisance parameters
(which determine the calibration vector and the kernel function),
$\Delta$ is a residual column vector,
$f(\alpha)$ is a calibration column vector,
the ``$\ast$'' operator is the pixel-wise multiplication operator,
$C$ is the Gaussian Process covariance matrix,
made up of $N^2$ values $C_{nn'}$,
$K_\alpha$ is a kernel function,
and each $\lambda_n$ is the wavelength of spectral pixel $n$.
The calibration vector $f(\alpha)$ is going to take care of the
large-scale calibration residual; the Gaussian process is going to
pick up smaller-scale residuals away from \emph{that}.

In detail, we are going to choose a $M$th-order polynomial form for
the calibration vector $f(\alpha)$ and a bell-shape (Gaussian) for the
kernel function $K_\alpha(\cdot)$:
\begin{eqnarray}
f(\alpha) &=& \sum_{m=1}^M c_m\,x^m
\\
x &\equiv& \frac{\lambda}{\lambda_0} - 1
\\
K_\alpha(x) &=& a^2\,\exp(-\frac{x^2}{2\,\ell^2})
\\
\transpose{\alpha} &\equiv& \left[ \{c_m\}_{m=0}^M, a^2, \ell^2 \right]
\quad ,
\end{eqnarray}
where $\lambda$ is a column vector made up of the wavelengths $\lambda_m$
and $x$ is a scaled and shifted version of the same,
scaled by a fiducial wavelength $\lambda_0$.
Nuisance parameters $c_m$ set the shape of the large-scale calibration
issues, while $a^2$ and $\ell^2$ set the variance and correlation
length of the remaining covariant calibration noise.

\section{Experiments and results}

BDJ INSERT WORDS HERE.

\section{Discussion}

BDJ INSERT WORDS HERE.

HOGG COMPUTE TOTAL ANNUAL COST OF SPECTROPHOTOMETRY (to astronomical community).

All the code used in this project is available under an open-source license
  from \url{http://github.com/bd-j/something/}.
This version of the paper was generated
  from a git repository available at \url{http://github.com/bd-j/speccal/}
  with git hash \texttt{\githash\,(\gitdate)}.

\acknowledgements
Thanks to X. for bringing us beer.

\end{document}

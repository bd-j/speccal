\documentclass[iop,numberedappendix]{emulateapj}
\usepackage{apjfonts}
\bibliographystyle{apj}

\input{vc}

\begin{document}

\title{Combining information from photometric and spectroscopic data:\\
  Don't waste your time in spectral calibration!}
\author{BDJ, DRW, DFM, DWH, CC, others}
\affil{foo}

\begin{abstract}
In a typical spectral-fitting project,
  there tend to be a small number of bands of well-calibrated photometry
  and thousands of poorly calibrated spectroscopic pixels.
Good spectroscopic calibration is both expensive and difficult,
  but naive weighted least-square fitting will weight
  the thousands of spectroscopic pixels far higher overall
  than the few photometric bands.
Here we present a flexible and general spectroscopic calibration model,
  which can be fit simultaneously along with whatever spectral parameters.
The model involves a polynommial calibration vector to capture large-scale calibration issues
  and a Gaussian Process to capture small-scale wiggles.
We show that, in this framework, the quality of some kinds of spectral parameter fits
  is not a strong function of the quality of the spectroscopic calibration;
  that is, for many scientific goals there is no scientific reason
  to obtain good spectrophotometric calibration.
In particular, for stellar population fitting, high signal-to-noise data,
  and good photometry,
  we show that raw counts are just as good as calibrated counts for measuring the
  parameters of interest.
\end{abstract}

\keywords{
  greetings
  ---
  whole world
}

\section{Spectral calibration is a bitch}

Hello World.

\section{Model generalities}

Hello World.

\section{Experiments and results}

Hello World.

\section{Discussion}

All the code used in this project is available under an open-source license
  from \url{http://github.com/bd-j/something/}.
This version of the paper was generated
  from a git repository available at \url{http://github.com/bd-j/speccal/}
  with git hash \texttt{\githash} (\gitdate).

\acknowledgements
Thanks to X. for bringing us beer.

\end{document}
